%
%  untitled
%
%  Created by Tristan Himmelman on 2008-05-06.
%  Copyright (c) 2008 __MyCompanyName__. All rights reserved.
%
\documentclass[]{article}

% Use utf-8 encoding for foreign characters
\usepackage[utf8]{inputenc}

% Setup for fullpage use
\usepackage{fullpage}

% Uncomment some of the following if you use the features
%
% Running Headers and footers
%\usepackage{fancyhdr}

% Multipart figures
%\usepackage{subfigure}

% More symbols
%\usepackage{amsmath}
%\usepackage{amssymb}
%\usepackage{latexsym}

% Surround parts of graphics with box
\usepackage{boxedminipage}

% Package for including code in the document
\usepackage{listings}

% If you want to generate a toc for each chapter (use with book)
\usepackage{minitoc}

% This is now the recommended way for checking for PDFLaTeX:
\usepackage{ifpdf}

%\newif\ifpdf
%\ifx\pdfoutput\undefined
%\pdffalse % we are not running PDFLaTeX
%\else
%\pdfoutput=1 % we are running PDFLaTeX
%\pdftrue
%\fi

\ifpdf
\usepackage[pdftex]{graphicx}
\else
\usepackage{graphicx}
\fi
\title{Aruspix Development Set up}
\author{  }

\begin{document}

\ifpdf
\DeclareGraphicsExtensions{.pdf, .jpg, .tif}
\else
\DeclareGraphicsExtensions{.eps, .jpg}
\fi

\maketitle

Three libraries must be downloaded and compiled in order to be able to compile Aruspix. 
They are the following:
\begin{itemize}
\item wxWidgets 2.8.3 (http://www.wxwidgets.org/)
\item IM (http://www.tecgraf.puc-rio.br/im/) 
\item Torch(http://www.torch.ch/)
\end{itemize}

\section{wxWidgets Compilation}
	
Note: wxWidgets must be compiled using the MaxOSX 10.4u sdk to enable Aruspix use with both Mac OS 10.4 and 10.5.
\begin{enumerate}
	\item Create two directories within the wxWidgets directory: osx-static, osx-static-debug.
	\item In the osx-static-debug directory run the following commands to compile for debug mode:
	\begin{verbatim} ../configure --disable-shared --enable-debug --with-libjpeg=builtin --with-libpng=builtin 
     	--with-macosx-sdk=/Developer/SDKs/MacOSX10.4u.sdk
 		clean
 		make
	\end{verbatim}

	\item In the osx-static directory run the following commands to compile for release mode:
	\begin{verbatim} ../configure --disable-shared --enable-universal_binary --with-libjpeg=builtin
	     --with-libpng=builtin --with-macosx-sdk=/Developer/SDKs/MacOSX10.4u.sdk 
	 clean
	 make
	\end{verbatim}
\end{enumerate}

\section{Aruspix Compilation}

\subsection{Setting Aruspix Environment Variables}
	In the aruspix/osx directory you will find an XML file named enviroment.plist. This file is used by Xcode to set the following environment variables for linking purposes:
	\begin{itemize}
		\item \begin{verbatim}ARUSPIX: Location of Aruspix project directory.\end{verbatim}
		\item \begin{verbatim}ARUSPIX_IMLIB: Location of IM library.\end{verbatim}
		\item \begin{verbatim}ARUSPIX_TORCH: Location of Torch library.\end{verbatim}
		\item \begin{verbatim}ARUSPIX_WX: Location of wxWidgets library.\end{verbatim}
		\item \begin{verbatim}ARUSPIX_WX_VERSION: wxWidgets library version number.\end{verbatim}
	\end{itemize}

	environment.plist prototype:
	\begin{verbatim}
		<?xml version="1.0" encoding="UTF-8"?>
		<!DOCTYPE plist PUBLIC "-//Apple//DTD PLIST 1.0//EN" "http://www.apple.com/DTDs/PropertyList-1.0.dtd">
		<plist version="1.0">
		<dict>
		    <key>ARUSPIX</key>
		    <string>/Users/puginl/projects/aruspix</string>
		    <key>ARUSPIX_IMLIB</key>
		    <string>/Users/puginl/libs/imlib</string>
		    <key>ARUSPIX_TORCH</key>
		    <string>/Users/puginl/libs/Torch3</string>
		    <key>ARUSPIX_WX</key>
		    <string>/Users/puginl/libs/wx2.8.7</string>
		    <key>ARUSPIX_WX_VERSION</key>
		    <string>2.8</string>
		</dict>
		</plist>
	\end{verbatim}
	\begin{enumerate}
		\item You must modify the paths within this file to match your directory structure.
		\item In the home directory create the following hidden directory: .MacOSX.
		\item Now copy the modified environment.plist file into the .MacOSX directory.
		\item You must now log out and log back in.
	\end{enumerate}

\subsection{Compilation of Machine Learning Executables used by Aruspix}
	\begin{enumerate}
		\item Open aruspix/osx/torch.xcodeproj with Xcode.
		\item Compile the following executables in both release and debug mode: adapt, decoder, ngram.
	\end{enumerate}

Open aruspix/osx/aruspix.xcodeproj with Xcode.

Now Aruspix is ready to be compiled in both Debug and Release mode. 
\end{document}
