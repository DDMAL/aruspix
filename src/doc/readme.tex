%
%  Aruspix Readme
%  Created by Tristan Himmelman on 2008-06-03.
%
\documentclass[]{article}

% Use utf-8 encoding for foreign characters
\usepackage[utf8]{inputenc}

% Setup for fullpage use
\usepackage{fullpage}

% Surround parts of graphics with box
\usepackage{boxedminipage}

% Package for including code in the document
\usepackage{listings}

% If you want to generate a toc for each chapter (use with book)
\usepackage{minitoc}

% This is now the recommended way for checking for PDFLaTeX:
\usepackage{ifpdf}
\usepackage{url}

%\newif\ifpdf
%\ifx\pdfoutput\undefined
%\pdffalse % we are not running PDFLaTeX
%\else
%\pdfoutput=1 % we are running PDFLaTeX
%\pdftrue
%\fi

\ifpdf
\usepackage[pdftex]{graphicx}
\else
\usepackage{graphicx}
\fi
\title{Aruspix Development Set Up}
\author{  }

\begin{document}

\ifpdf
\DeclareGraphicsExtensions{.pdf, .jpg, .tif}
\else
\DeclareGraphicsExtensions{.eps, .jpg}
\fi

\maketitle

\section{Introduction} 
Aruspix uses four libraries that have to be installed on your system for it to compile. 
They are the following:
\begin{itemize}
	\item wxWidgets 2.8.7 (http://www.wxwidgets.org/)
	\item IM 3.1 for VC 6 (http://www.tecgraf.puc-rio.br/im/) 
	\item Torch 2.0 (http://www.torch.ch/)
	\item TinyXML (http://www.grinninglizard.com/tinyxml/index.html/)
\end{itemize}
This document will outline the steps involved in compiling these libraries 
for Windows XP and MacOS X operating systems.

For MacOS X, TinyXML is include in the project and does not need to be installed and compiled separately.

%*****************************************************************************
\section{Windows XP Set Up (Using Microsoft Visual C++ 6)}
%*****************************************************************************
Note: Microsoft Visual C++ 6 has been the main Windows development tool up to 
date. However the project has also been compiled and run successfully using
Microsoft Visual C++ 8 following the same set up guide. Some of the project
properties may need to be adjusted as they are not always maintained when
upgrading a project. Refer to the trouble shooting section if problems arise.
%*****************************
\subsection{wxWidgets}
To compile wxWidgets open \begin{verbatim} wxWidgets-2.8.7\build\msw\wx.dsw \end{verbatim}
using VC 6. Select Batch Build from the Build drop down menu. Now select the
following three configurations from each section and build:
\begin{itemize}
    \item Win32 DLL Debug
    \item Win32 Release
    \item Win32 Debug
\end{itemize}	

%*****************************
\subsection{IM Library}

This library is available pre-compiled. Just ensure that the library you've
downloaded has a file structure that matches the following example:
\begin{verbatim}
\imlib
    \include : should contain all source files
    \lib : should contain all library files
\end{verbatim}
 
%*****************************
\subsection{TinyXML}

To compile this library simply open \begin{verbatim} tinyxml\tinyxml.dsw \end{verbatim}
with VC 6. Select Batch Build from the Build drop down menu and the build the
whole project. Now ensure that the file structure matches the following example:
\begin{verbatim}
\tinyxml : Contains the tinyxml source files.
    \Debug : Contains the debug tinyxml lib file.
    \Release : Contains the release tinyxml lib file.
\end{verbatim}http://stackoverflow.com/questions/5476840/can-you-explain-theese-disturbing-anomalies-of-md5-and-modulo
	
%*****************************
\subsection{Torch}

After downloading Torch, follow the instructions provided on the Torch website
for compiling the library on Windows. Be sure to name the project "Torch3".
\begin{verbatim}http://www.torch.ch/matos/w_install.pdf \end{verbatim}
After compilation, make sure the Torch library folder has a file structure
that matches the following example:
\begin{verbatim}
\Torch3
    \Debug : Contains the Torch3.lib (Debug) file
    \Release : Contains the Torch3.lib (Release) file
    \src : Contains the Torch source files
\end{verbatim}
Next you need to add the following environment variable to your system: 
\begin{itemize}
	\item TORCH : Path to the Torch folder. 
\end{itemize}
To do this, right click on the My Computer icon and select properties. Then
select Environment Variables from the Advanced tab. Add the variable to the
"User variables" section.\\\\
Now open \begin{verbatim} aruspix\win32\torch.dsw \end{verbatim} with VC 6.
Select batch build and build all the project configurations.
  
%*****************************
\subsection{Aruspix}

To compile Aruspix the following environment variables must added to operating
system's configuration:
\begin{itemize}
	\item IMLIB : Path of the IMlib folder
	\item TINYXML : Path of the TinyXML folder
	\item WXWIN : Path of the wxWidgets folder
	\item WXWIN\_VERSION : wxWidgets version number (format ex. for 2.8.7: 28)
\end{itemize}
Now simply open aruspix/win32/Aruspix.dsw and you should be ready to compile Aruspix.

\subsection{Troubleshooting}

If linking errors occur at any stage in the setup it is likely due to the code
generation settings of the projects. To fix the problem ensure that all the
libraries and projects are compiled using the same code generation settings
which can be found in: 
\begin{verbatim} project settings/properties -> C/C++ -> Code Generation -> Use run-time library \end{verbatim} 
We have been using "Multithreaded DLL" for release versions and "Multithreaded
Debug DLL" for debug versions.

%*****************************************************************************
\section{Mac OS 10 Set Up (Using Xcode)}
%*****************************************************************************

%******************************
\subsection{wxWidgets Compilation}
	
\begin{enumerate}
	\item Create two directories within the wxWidgets directory: osx-static, osx-static-debug.
	\item In the osx-static-debug directory run the following commands to compile for debug mode:
	\begin{verbatim} ../configure --disable-shared --enable-debug --with-libjpeg=builtin --with-libpng=builtin 
     	--with-macosx-sdk=/Developer/SDKs/MacOSX10.5.sdk
 		make clean (if necessary)
 		make
	\end{verbatim}

	\item In the osx-static directory run the following commands to compile for release mode:
	\begin{verbatim} ../configure --disable-shared --enable-universal_binary --with-libjpeg=builtin
	     --with-libpng=builtin --with-macosx-sdk=/Developer/SDKs/MacOSX10.5.sdk 
	 make clean (if necessary)
	 make
	\end{verbatim}
	\item On OSX 10.6 you may need to add beforehand:
	\begin{verbatim}
	export arch_flags="-arch i386"
	../configure CFLAGS="$arch_flags" CXXFLAGS="$arch_flags" \
             CPPFLAGS="$arch_flags" LDFLAGS="$arch_flags" ...
	\end{verbatim}
	if you get an error while compiling Wx.
\end{enumerate}


%******************************
\subsection{Setting Aruspix Environment Variables}
	In the aruspix/osx directory you will find an XML file named sample\_enviroment.plist. This
	file is used by Xcode to set the following environment variables for linking purposes:
	\begin{itemize}
		\item \begin{verbatim}ARUSPIX_IMLIB: Location of IM library.\end{verbatim}
		\item \begin{verbatim}ARUSPIX_TORCH: Location of Torch library.\end{verbatim}
		\item \begin{verbatim}ARUSPIX_MEI: Location of libmei framework.\end{verbatim}
		\item \begin{verbatim}ARUSPIX_GTEST: Location of gtest framework.\end{verbatim}
		\item \begin{verbatim}ARUSPIX_WX: Location of wxWidgets library.\end{verbatim}
		\item \begin{verbatim}ARUSPIX_WX_VERSION: wxWidgets library version number.\end{verbatim}
	\end{itemize}

	environment.plist prototype:
	\begin{verbatim}
		<?xml version="1.0" encoding="UTF-8"?>
		<!DOCTYPE plist PUBLIC "-//Apple//DTD PLIST 1.0//EN" "http://www.apple.com/DTDs/PropertyList-1.0.dtd">
		<plist version="1.0">
		<dict>
		    <key>ARUSPIX_IMLIB</key>
		    <string>/Users/puginl/libs/imlib</string>
		    <key>ARUSPIX_TORCH</key>
		    <string>/Users/puginl/libs/Torch3</string>
			<key>ARUSPIX_MEI</key>
			<string>/Users/jamie/ax-libs/osx</string>
			<key>ARUSPIX_GTEST</key>
			<string>/Users/jamie/ax-libs/osx</string>
		    <key>ARUSPIX_WX</key>
		    <string>/Users/puginl/libs/wx2.8.7</string>
		    <key>ARUSPIX_WX_VERSION</key>
		    <string>2.8</string>
		</dict>
		</plist>
	\end{verbatim}
	\begin{enumerate}
		\item You must modify the paths within this file to match your directory structure and rename it environment.plist.
		\item In your home directory create the following hidden directory: .MacOSX.
		\item Now copy the modified environment.plist file into the .MacOSX directory.
		\item You must now log out and log back in.
	\end{enumerate}


\subsection{libmei}
Do the same as gtest with libmei (copy the framework somewhere, add existing framework, change DYLD\_FRAMEWORK\_PATH)
	
%******************************
\subsection{Compilation of Machine Learning Executables used by Aruspix}
	\begin{enumerate}
		\item Open aruspix/osx/torch.xcodeproj with Xcode.
		\item Compile the following executables in both release and debug mode: adapt, decoder, ngram.
	\end{enumerate}
	
	From version 2.2, Julius recognizer is used. It needs to be compiled separately from the command line. Copy the Makefile\_sample to Makefile and change it appropriately (set the path to Julius). Julius version 4.2.2 or greater is necessary.
	
	Then run:
	\begin{verbatim} 
		make
		make install
	\end{verbatim}

%******************************
\subsection{Compilation of Aruspix}
Open aruspix/osx/aruspix.xcodeproj with Xcode.
Aruspix should be ready to be compiled in both Debug and Release mode. 

\section{Linux Set Up}

\subsection{Dependencies}
 \begin{verbatim}
	sudo apt-get install libtorch3-dev libwxgtk2.8-dev wx-common libimlib2-dev libxml2-dev cmake
 \end{verbatim}

 IMlib is available from \url{http://www.tecgraf.puc-rio.br/im/}
\subsection{Compilation}

 \begin{verbatim}
	mkdir build
	cd build
	cmake ../linux
	make
 \end{verbatim}

%*****************************
\section{Aruspix Source Code Organization}

This file describe the source code organization of the project, with the directory tree.
A first group of directories contains files that are common to the application, as the
other directories are specific to the different workspace environment. Every workspace
environment is contained into a single directory that can be excluded from the application
by changing the preprocessor definitions.\\

\subsection{Common files}

/app : This directory contains the files that constitute the core of the application.
The are prefixed with "ax". Most of the a related to GUI classes.\\\\
/im : This directory contains the files related to image processing. Classes and
function that extend some functionalities of the IMLIB image processing library,
such as advanced binarization methods, are to be find here.\\\\
/ml : This directory contains the files that relate to machine learning.\\\\
/mus : This directory contains the files that implement the music editor and handle the music files.\\\\

\subsection{Workspace environment files}

Workspace environments can be enabled or disabled at compilation by changing
preprocessor definitions. The files generated with wxDesigner cannot be disabled
by changing the preprocessor definitions and have to be removed by hand (this
has to be fixed).\\\\
/empty : This directory contains a workspace template (May be out of date).\\\\ 
/superimposition : This directory contains the files that implement the superimposition
of digital music on to the original images.\\\\
/recognition : This directory contains the files that deal with the music recognition.\\\\
/comparison :

\end{document}
